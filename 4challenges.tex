\chapter{Research Challenges}
\label{sec:challenges}

\section{Challenges associated with Formal reasoning}

To understand the complexities involved in mechanical
certification of an algorithm that was not designed
originally with certification in mind, we need to re-visit
the general approach to applying formal reasoning on
software programs.  The typical approach is to break the
program into a number of pieces, prove key lemmas
characterizing the role of each piece, and then chain these
lemmas together into a proof of the correctness of the
entire program. Crucial to this approach, however, is the
requirement that each program piece can be characterized by
a succinct invariant that can be easily verified.  However,
in a program not developed with reasoning in mind,
optimizations typically destroy the structural disciplines
and modularity of the individual program pieces. This makes it
difficult to identify and isolate the components that
actually maintain succinct, interesting invariants.

For instance, to prove the correctness statement in the previous algorithm, 
we want to prove that the complete algorithm follows
the invariant that the execution of input CCDFG is equal to execution of the output CCDFG.
Since the algorithm is composed of four concrete steps -- generate scheduling steps, add shadow register,
add edges and data propagation, we intuitively expect the individual steps or at least a combination of steps in sequence to
follow this invariant. However, since the algorithm has not been designed keeping theorem proving
in mind, that is not the case. For example, if
we consider the first step of the proposed algorithm -- {\bf generating new scheduling steps} by
overlapping executions of an unrolled loop, we know that the execution 
of the sequential scheduling steps is not the same as the execution of new scheduling steps
unless we prove that there are no data hazards. But, data hazards are not completely eliminated till the last step of the algorithm. Note, that the complete algorithm does follow the invariant as expected, but reasoning about the structure of the
complete algorithm at once is not easy.

Our first approach was to certify their implementation as it is using theorem proving. 
But, our experience was that it is a difficult approach, one that we need not endure.
In general, in order to certify such an arbitrary implementation,
one has to either (1)~restructure the implementation into
one that is more disciplined, and prove the equivalence
between the two, or (2)~come up with very complex
invariants that essentially comprehend how invariants from
each individual piece are conflated together in the
implementation.  Both approaches require extensive human
interaction, resulting in the proverbial euphemism of proofs
of programs being orders of magnitude more complex than the
programs themselves~\cite{liu}.
%In addition, the previous algorithm does not take a CCDFG as input, but rather
%works with microsteps, edges and schedule of a CCDFG. That can get tricky because to analyze the
%semantic run of a CCDFG, we have to now simultaneously analyze all the three inputs. In our approach,
%we work with the CCDFG itself and define the run of a CCDFG semantically.

%Furthermore, the previous algorithm initially unrolls the loop and later adds
%a back edge to mimick the pipeline full stage. Unrolling loop with unique names
%for each microstep makes it easier to design the algorithm but formally we lose the notion
%that each microstep is part of a loop and not different in each iteration.
%Besides, unrolling loop makes it very difficult to reason for the correctness of the
%back edge in the
%full pipeline stage.

In our work, however, we can ``get away'' without verifying
the specific implementation while still being able to
certify the design generated by behavioral synthesis without
loss of fidelity. The key observation, as above, is that it
is sufficient to develop {\em any} certifiable algorithm
that generates a pipelined CCDFG from a sequential
implementation which can be effectively applied with SEC.
In particular, any certifiable algorithm that has the same
input-output characteristic as the proposed algorithm
is sufficient.  Thus, our dissertation is on identifying
certifiable primitives and invariants of a loop pipelining
transformation and developing a pipeline generation
algorithm using those primitives, achieving the dual goal of
mechanical reasoning of the algorithm and amenability of the
resulting reference model to SEC.

\section{Importance of using Formal Methods for checking correctness}

Formally certifying an algorithm gives confidence that the pipelined design is indeed correct. We can claim that if a pipeline loop is created, then there are no additional data hazards which have not been accounted for. Also, since our final theorem proves that executing a sequential loop is same as executing the pipelined loop generated from our algorithm, we can confidently say that our algorithm is complete and data and control flows are well-maintained.  

Note that our framework is independent of the inner workings of a specific tool, and can be applied to certify designs synthesized by different tools from a broad class of ESL descriptions. Also, the approach produces a certified reference flow, which makes explicit generic invariants that must be preserved by different transformations. Checking correctness using formal methods prompted us to address the issues lacking in the previous algorithm. To ensure that control flow is maintained, we had to deal with branches. The previous algorithm introduces the concept of Exit edges but does not explain/implement them. The previous authors checked the output of their algorithm with RTL under the assumption that the loop never exits, hence they did not face any issue while testing. However, removing a conditional branch in a loop and furthermore, adding the conditional branch back in the middle of a pipelined loop requires complex reasoning which we manage using one of our primitives, explained in Chapter~\ref{sec:pipelining-algorithm}.

Also, the invariant that data flow is maintained at each step enabled us to find a bug in the previous algorithm. The previous algorithm moves a statement to make sure one particular data hazard is removed, but in doing so they move the statement across a conditional branch statement. Our primitves ensure that such a move is not possible. We have restructred the data propagation step so that instead of going across a conditional branch in the same iteration, the movement of step is now to the previous iteration, explained in Chapter ~\ref{sec:pipelining-algorithm}. 

%Furthermore, there is a different mindset required when we merely write an algorithm Vs when we want to prove an algorithm using theorem proving. For example, adding a shadow register step may look like a trivial step which requires simple reasoning about read and write variables, but when we have to mathematically prove that such a step maintains the control and data flow, we have to reason about the fact that the new variable is indeed a new variable not introduced anywhere else in the algorithm. 


