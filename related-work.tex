\chapter{Related Work and Novelty of Our Approach}
\label{chapter:related-work}

Besides behaviorally synthesized pipelines, there are mainly two other kinds of pipelines, hardware pipelines and software pipelines. 

\section{Hardware Pipelines and Their Verification}

Hardware pipelining~\cite{Hennessy-2003} is of two types: {\em Instruction pipelining} where there is a continuous, 
overlapped movement of instructions to the processor or {\em Arithmetic pipelining} where 
different stages of an arithmetic operation are handled along the stages of a pipeline.
An Instruction pipeline has five stages: Fetch, Decode, Execute, Memory Access 
and Write Back. Without any pipelining, 
a processor gets the first instruction from memory, undergoes arithmetic operations 
and then sends it back to memory before starting any new instruction. While pipeline is 
fetching the instruction, Arithmetic and Logic Unit (ALU) of processor is idle. 
Pipelining allows the fetching of instructions to be continuous. The next 
instructions can be fetched even while the 
processor is performing arithmetic operations, holding them in a buffer close 
to the processor until each instruction operation can be performed. 
It reduces the processing time. However, there may be
hardware conflicts (structural hazards), data dependencies (data hazards) or hazards that come from 
branch, jump and other control flow changes (control hazards). These prevent the pipeline from
running at full speed. These issues can and are successfully dealt with. But,
detecting and avoiding these hazards leads to a considerable increase in hardware complexity. 

There has been a significant amount of work in formal or semi-formal verification of processor (hardware) pipelines. 
A theorem prover, PVS~\cite{pvs} 
was successfully used for verification of a simple pipelined processor~\cite{Cyrluk94}. Sawada and Hunt~\cite{sh:pipeline} 
presented a technique that models the trace of executed 
instructions using a table-based representation called a MAETT. These approaches require involvement 
of user to a great degree, especially in control dominated designs. Hosabettu~\cite{Hosabettu2000} proposed to build the proof of correctness 
of pipelined microprocessors by constructing the abstraction function 
using completion functions. Burch and Dill~\cite{bd:pipeline} presented a technique to verify the 
correctness of the implementation model of a pipelined processor 
against its instruction-set architecture (ISA) model based on 
quantifier-free logic of equality with uninterpreted functions. 
The technique has been extended to handle more complex pipelined
architectures by several researchers~\cite{Skakkebæk1998,Velev2000,VelevBryant2000, pm:pipelines, velev05,sud-2010}. 
%The approach~\cite{VelevBryant2000}focuses on effi-
%ciently checking the commutative condition for complex microarchitectures by
%reducing the problem to checking equivalence of two terms in a logic with equality,
%and uninterpreted function symbols. 
ARM2 pipelined processor was verified~\cite{Huggins1998} using abstract state machine. Levitt and Olukotun~\cite{levitt1997} created a verification method to merge repeatedly last two stages of a pipeline into one, called unpipelining, to ultimately create a sequential verision. 
Aagaard et al.~\cite{Aagaard2001} presented a framework
for microprocessor correctness statements about safety that is independent
of implementation representation. Out of order pipelines have been verified by combining model checking for the verification 
of the pipeline control, and theorem proving for the verification of the pipeline functionality~\cite{Jacobi2002} . 

All the above techniques attempt to formally verify the implementation of
pipelined processors by comparing the pipelined implementation with its sequential
(ISA) specification model, or by deriving the sequential model from
the implementation. There are significant differences in goals
and techniques between these efforts and ours.
Microprocessor pipelines include optimized (hand-crafted)
control and forwarding logics, but have a static set of
operations based on the instruction set. Behaviorally synthesized 
loop pipelines tend to be deep with a high complexity at each stage, but
control and forwarding logics are more standardized since
they are automatically synthesized. 
Furthermore, microprocessor pipeline verification
is focused on one (hand-crafted) pipeline implementation,
while our work focuses on verifying an {\em algorithm
that generates pipelines}. As explained earlier in Chapter 6  that our invariant is very different from a typical invariant
used in the verification of pipelined machines (\eg, for
microprocessor pipelines). We make explicit the
correspondence with the sequential execution. The key
requirement from a pipeline invariant, \viz, hazard freedom,
is left implicit and arises indirectly as a proof obligation
for invariance of this predicate. 
 
\section{Software Pipelines and Their Verification}
Software pipelining is a form of out of order execution. It is performed by compiler rather than a processor. 
Behaviorally synthesized loop pipelines are similar in reasoning to software loop pipelines except that since
behavioral synthesis is automatic, it is much more streamlined than software pipelines. 

In~\cite{Leviathan:2002}, Pnueli and Leviathan present a validator to verify software 
pipeline in IA-64 architecture~\cite{Doshi:1999} (Intel's 
architecture specifically designed with keeping complexities of software pipelining in mind and to provide additional 
support for it). It uses rotating register file and predicate registers for its verification. 
Using symbolic evaluation, the validator generates a set of verification conditions 
that are discharged by a theorem prover. Kundu et al.~\cite{Kundu:2009} 
use parameterized translation validation to verify software pipelines. They use 
code motion (rewrite of original loop by validating each rewrite step).
%Goldberg et al.~\cite{Goldberg} use run-time checks to verify behaviors 
%of pipelined loop but they did not verify full semantic preservation.
Our understanding of hazards and reasoning behind pipelining algorithm 
is very closely related to recent work on
verification of software pipelines.  In particular, Tristan
and Leroy~\cite{tl:software-popl10} present a verified
translation validator for software loop pipelines.  The loop
pipelines in behavioral synthesis considered in this paper
are close in structure to software loop pipelines, although
our formalization (\eg, CCDFG) has different semantics from
the Control Flow Graphs they use, reflecting the difference
between eventual targets of compilation (\viz, hardware
vs. software).  However, the fundamental difference is in
the approach taken to actually certify the pipelines.
Tristan and Leroy's approach decomposes the certification
problem into two parts, a ``dynamic'' part that is certified
on a case-by-case basis  and a ``static'' part that is
certified in the Coq theorem prover~\cite{coq} once and for all.  The
theorem proven by Coq is informally paraphrased as follows:

\begin{quote}
Suppose the pipelining algorithm generates a pipeline
${\cal{P}}$ from a sequential design ${\cal{S}}$.  Suppose
symbolic simulation of ${\cal{S}}$ and ${\cal{P}}$ verifies
certain ``dynamic'' verification conditions (VCs).  Then
${\cal{S}}$ and ${\cal{P}}$ are indeed semantically
equivalent.
\end{quote}

\noindent
Thus for any pipeline instance ${\cal{P}}$ generated by
their algorithm, symbolic simulation is executed between
${\cal{P}}$ and ${\cal{S}}$ to certify that ${\cal{P}}$ is
indeed a correct pipelined implementation of ${\cal{S}}$.
The dynamic VCs checked by symbolic simulation essentially
certify that the pipeline generation did not overlook any
hazards.

This is where our work differs from theirs.  Our work is
expected to provide a single theorem certifying the
correctness of the reference pipelined implementation,
without requiring further runtime hazard check.
Furthermore, their correspondence theorem relates the
pipelined implementation with a sequential design with a
(bounded) unrolled loop, while our approach certifies the
correspondence between the actual Control Flow Graph (CFG)
and the pipelined implementation.  Indeed, Tristan and Leroy
remark that the mechanization of the correspondence between
the CFG and unrolled loop is ``infuriatingly difficult''.
We speculate this is so because they focus on verifying the
correspondence between the unrolled loop and the pipeline.
In our experience, attempting the formal correspondence between the unrolled
sequential loop and pipelined design is indeed difficult since
there is no formal way to connect to back edge of the loop
with any of the edges in the pipeline.  We believe that
reconciling this problem and developing a fully certified
pipeline generation algorithm would require backtracking
from the correspondence with an unrolled loop (and hence
translation validation) to a more complex invariant like
ours.  Of course we must note that we can ``afford'' to
develop a fully certified algorithm in our approach since
the pipelines are simpler
(cf. Chapter~\ref{sec:formalization}); achieving this for
arbitrary software pipeline may require further more subtle
invariants.

\section{Verification of Behaviorally synthesized designs}

A lot of research has been done in verification of behaviorally synthesized designs. Matsumoto et al.~\cite{Matsumoto} compare two similar C-based hardware descriptions. 
To verify large C descriptions efficiently, they rely on scanning for textual differences 
to reduce problem complexity, then enumerate execution paths and 
apply symbolic simulation and word-level uninterpreted functions. 
Bounded model checking is used if the software is abritrary. 
If the software is arbitrary high-level code, then full formal verification
is undecidable, but bounded-length verification is possible using symbolic execution.
Kroening, Clarke and Yorav~\cite{Kroening1} apply BMC (Bounded Model
Checking) to both a circuit and a C program. Their tool covers arbitrary designs. However,
this method shows only the absence of inconsistencies up to a given bound.
Furthermore, the number of paths is very high. 
In order to avoid the state space explosion problem of full formal verification,
Jain, Kroening, and Clarke~\cite{Clarke:2005} introduce 
predicate abstraction for hardware implementations
against software specifications. This approach can greatly reduce the
size of the state space and verify certain properties for large circuits. The strength
of that work is powerful abstraction techniques that reduce the complexity of the
software specifications. However, such abstraction techniques can be too coarse, and
finding good predicates is highly challenging. 

Initially, high level synthesis verification focused on behavioral VHDL~\cite{19} and 
translation from VHDL to dependence flow graphs~\cite{21} was verified by structural induction
based on CSP semantics~\cite{22}. 
%Koelbl et al. (need to add citation) 
%provide a tutorial introduction on methods of comparing high-level designs with RTL. 
%Chauhan et al. (need to add citation) propose a technique for SEC between non-cycle-accurate designs by constructing a 
%pair of normalized cycle-accurate designs from the original designs. 
Bisimulation has been proposed as a solution to validate behaviorally synthesized designs~\cite{kundu2008}.  
Their approach is implemented for the Spark synthesis tool~\cite{spark}.
However, their approach is not scalable and we handle more complex industrial strength designs. 
%they do not provide a uniform design representation
%across different synthesis tools or implement optimizations
%necessary for scaling to design sizes that we handle.
Recently, HOL~\cite{23}has been used to synthesize hardware from formal
languages automatically.  A certified hardware
synthesis from programs in Esterel, a synchronous design language, has been also
been developed~\cite{24} in which a variant of Esterel was embedded in HOL.
%Dave [25] provides a comprehensive bibliography of compiler verification. One of
%the earliest work on compiler verification was the Piton project [26], which verified a
%simple assembly language compiler. Compiler certification forms a critical component
%of the Verisoft project [27], aiming at correctness of implementations of computing systems
%with both hardware and software components. The Verifix [28] and CompCert [29]
%projects have explored a general framework for certification of compilers for various C
%subsets [30, 31]. There has also been work on a verifying compiler, where each instance
%of a transformation generates a proof obligation discharged by a theorem prover [32].


There has been much research on sequential equivalence checking (SEC) between
RTL and gate-level hardware designs~\cite{33, 34}. Research has also be done on combinational
equivalence checking between high-level designs in software-like languages (e.g.,
SystemC) and RTL-level designs~\cite{hu}. There has also been effort for SEC between software
specifications and hardware implementations~\cite{36} .

%[22, 23, 61]. 

%Edmund et al.[8] present a way of using bounded model checking to verify the consistency
%of behaviours for C and Verilog programs. Given an ANSI-C program and a
%Verilog circuit, both are translated into a formula similar to an FSMD that represents
%behavioural consistency. The formula is then checked using SAT. Note that the ANSI-C%
%program and Verilog circuit have no HLS connections. In other words, to verify a circuit,
%one has to manually develop a specifically formatted C program that is functionally
%equivalent to the Verilog circuit.

\section{Use of Theorem Provers in Hardware Verification}
Thorem provers are widely used for hardware verification. 
HOL theorem prover~\cite{hol} has been used in 
several well-documented projects~\cite{cohn,graham}.
ACL2 is also used a lot in hardware 
verification~\cite{Russinoff,car2,car,Hardin,ray-abstracting,ray-connecting,ray-certification}. 
Our project is however somewhat different from the
traditional applications of theorem provers. 
First, since an over-arching goal is to exploit automatic
decision procedures, we use theorem proving primarily to
complement automated tools. Second, we eschew theorem
proving on inherently complex or low-level implementations.
Third, interactive theorem proving is acceptable for
one-time use, in certification of a transformation, 
but not as part of a methodology that
requires ongoing use in certification of each design. The
constraints are imposed by the the environment in which we
envision our framework being deployed: it may not be
possible to have a dedicated team of experts doing theorem
proving as full-time jobs. Finally, the loop pipelining transformation
we verify are proprietary to the synthesis tools. Therefore, our approach is
targeting verification of transformations
which are closed-source (and exceedingly complex), thus
making traditional program verification techniques unusable.
Our approach shows a novel way in which theorem
proving can be applied even under those constraints, in
concert with automated SEC.

In addition to technical contributions, we see our work as
providing an important methodological contribution enabling
use of theorem proving in situations where one needs to
certify the result of an implementation on which theorem
proving cannot be directly applied either because it is
closed-source or because it is highly complex: (1)~create a
reference implementation, perhaps using as much information
as available from the actual implementation, in our case
information about pipeline intervals, (2)~certify this
simpler reference implementation with theorem proving, and
(3)~develop an SEC framework to compare the result of the
reference implementation with that of the actual
implementation.  In addition to making theorem proving
applicable on industrial flows without requiring us to
certify industial implementations with their full
complexity, this approach permits adjusting the algorithm
(within limits) to suit mechanical reasoning while still
affording comparison with actual synthesized artifacts.  We
have made liberal use of this ``luxury'', \eg, we have
been continually redefining our superstep construction function
to facilitate proof of key structural lemmas of the
invariant before settling on the final version.  We believe similar approach is applicable in
other contexts and may provide effective use of theorem
proving within industrial verification flows.
