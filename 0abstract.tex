Due to the rapidly increasing complexity in hardware designs and competitive time to market trends in the industry, there is an inherent need to move designs to a higher level of abstraction. Behavioral Synthesis is the process of automatically compiling such Electronic System Level (ESL) designs written in high-level languages such as C, C++ or SystemC into Register-Transfer Level (RTL) implementation in hardware description languages such as Verilog or VHDL. However, the adoption of this flow is dependent on designers' faith in the correctness of behavioral synthesis tools. 

Loop pipelining is a critical transformation employed in behavioral synthesis process, and ubiquitous in commercial and academic behavioral synthesis tools. It improves the throughput and reduces the latency of the synthesized hardware. It is complex and error-prone, and a small bug can result in faulty hardware with expensive ramifications. Therefore, it is critical to certify the loop pipelining transformation so that designers can trust the behaviorally synthesized pipelined designs.
Certifying a loop pipelining transformation is however, a major research challenge
because there is a huge semantic gap between the input sequential design and the output pipelined implementation, making it infeasible to verify their equivalence with automated sequential equivalence checking (SEC) techniques.

Complex loop pipelining transformations can be certified by a combination of theorem proving and SEC: (1) creating a certified pipelining algorithm which generates a reference pipeline model by exploiting pipeline generation information from the synthesis flow ({\em e.g.}, the iteration interval of a generated pipeline) and (2) conduct SEC between the synthesized pipeline and this reference model. However, a key and arguably, the most complex component of this approach is the development of a formal, mechanically verifiable loop pipelining algorithm. There has been an attempt to create such a pipelining algorithm but it was not certified. Infact, our experience shows that it was incomplete and error prone. We show how to systematically construct such an algorithm, and carry out its verification using the ACL2 theorem prover. We propose a framework of certified pipelining primitives which we show are essential for designing pipelining algorithms. Using our framework, we build a certified loop pipelining algorithm. We also propose a key invariant in certifying this algorithm, which links sequential loops with their pipelined counterparts. This is unlike other invariants that have been used in proofs of microprocessor pipelines so far.

This dissertation provides a framework for creating certified pipelining algorithms utilizing a mechanical theorem prover. Using this framework, we have developed a certified loop pipelining algorithm. This certified algorithm is essential in the overall approach to certify behaviorally synthesized pipelined designs. We demonstrate the scalability and robustness of our algorithm on ESL designs across various domains. 
